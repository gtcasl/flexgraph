\section{Introduction}

The end of Dennard Scaling \cite{Dark-Silicon} has moved the focus of computer architecture designers towards power and energy efficient architectures. However, energy efficient solutions such as ASIC designs present a serious limitation in flexibility and production cycle. These constraints have pushed production data centers towards using FPGA-based accelerators \cite{Catapult} for their reconfigurability and energy savings when compared to general-purpose graphics processing units (GPUs).
This trend has led to emergence of heterogeneous CPU-FPGA computing platforms \cite{Intel-FPGA} \cite{IBM-FPGA}, enabling the design of new energy efficient FPGA accelerators for domain specific applications. Graph Analytics is one the largest applications in production data centers today \cite{BigData}, its performance surfers from workload imbalance, frequent updates, limited data locality and low compute communication ratio making it memory-bound and energy inefficient. This problem is further exacerbated with the ever increasing size of its dataset, presenting a scalability challenge for the industry. Several solution have been proposed to address this problem both at the software level with better algorithms and programming abstraction \cite{GraphX} \cite{Galois} \cite{GraphMat} \cite{Pregel} \cite{GraphLab}, and at the hardware level with custom accelerators \cite {Graphicionado} \cite{Tesseract} \cite{GraphOps}. GraphMat \cite{GraphMat} and many other graph frameworks \cite{GraphMat} \cite{Pregel} \cite{GraphLab} define a vertex programming model to represent graph computation, but GraphMat is unique in that only the edges transformation of the kernel is accelerated, allowing the programmer to implement its algorithm is C++, instead of a proprietary language, increasing both accessibility and productivity. Flexgraph's design goals are aiming into achieving the same balance between efficiency and productivity via hardware acceleration. We developed FlexGraph using the Cocoh's \cite{Cocoh} framework, a C++ domain specific library for hardware design and simulation, providing a high level programming abstraction with the efficiency of native RTL. Several hardware description languages and libraries \cite{CHDL} \cite{Chisel} \cite{SystemC} \cite{BlueSpec} \cite{MyHDL} \cite{JHDL} \cite{Esterel} have been proposed as alternative to native verilog or VHDL to improve developer productivity by providing a higher level programming abstraction \cite{Chisel} \cite{SystemC} and execution model \cite{BlueSpec} \cite{Esterel}. Cocoh's unique strength is that it provides a uniform development and simulation environment based on the same source. Cocoh's modules are written directly in standard C++, debugged, tested and simulated using the same source, and later exported to verilog for FPGA deployment. Another strength of Cocoh's API is its integration with CHDL \cite{CHDL}, allowing gate-level simulation and analysis of the design. Using the Cocoh's API, developpers can write their custom GraphMat kernel for the FlexGraph accelerator and test their design before production. FlexGraph architecture uniquely targets commodity heteregenous CPU-FPGA computing platforms \cite{Intel-FPGA} for data centers. To the best of our knowledge it is a first implementation of a graph accelerator targeting this ecosystem. FlexGraphs's current performance is comparable to HLS, we have identified several performance optimizations that could further improve its performance which we will discuss later in the paper.
The remainder of this paper is organized as follows:
Section 2 describes GraphMat computing model and its SPMV kernel, section 3 describes FlexGraph's architecture, section 4 describes the experimental setup, section 5 describes our results and analysis, section 6 describes the related work, section 7 discusses future work and finally Section 8 summarizes our main contribution and results.