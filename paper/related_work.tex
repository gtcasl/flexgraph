\section{Related Work}

In this section, we discuss prior works related to this project. 

\subsection{GraphMat}

GraphMat \cite{GraphMat} is a high-performance Graph Analytics Framework for multi-core CPU platforms. It defines a C++ template-based vertex programming model for describing various graph algorithms in software and use Sparse Matrix Vector Multiplication (SPMV) as underlying compute kernel for efficient parallel processing. GraphMat also uses a doubly-compressed DCSC \cite{DCSC} matrix format for storage and compute efficiency. FlexGraph's programming model is an adaptation of GraphMat's model for collaborative computation  with an FPGA accelerator.

\subsection{Graphicionado}

Graphicionado \cite{Graphicionado} is a much recent project that is similar to FlexGraph. In this work, the authors implemented a graph accelerator targeting Intel Xeon platform. Graphicionado also uses a vertex programming model of graphs computation similar to GraphMat \cite{GraphMat}. It is a more specialized accelerator compared to FlexGraph in that it implements the entire graph algorithm in hardware, sacrificing flexibility for efficiency. Although the design allows some reconfigurability using FPGA, it still add many inconveniences compared to software. FlexGraph on the other hand only accelerate the graph edges computation in hardware, most of the algorithm is still written expressed in C++ and runs on the host processor like in GraphMat. FlexGraph reconfigurability is minimal and restricted to the SPMV module (changing the matrix datatype or the reduce operation). Graphicionado main architecture strength is their efficient optimizations of the graph access patterns to reduce stalls inside the pipeline.  

\subsection{GraphOps}

GraphOps \cite{GraphOps} is another graph accelerator also targeting FPGAs like FlexGraph. In this work, the authors propose an accelerator architecture that efficiently process a graph in memory encoded using a proposed locality-optimized scheme. Their main contribution is the graph storage representation which provides an efficient memory access pattern. A drawback of their proposal is that fact that the host processor has to pre-process the graph before the accelerator can consume it, which can add some considerable latency, considering that Graph Analytics algorithms tend to update the graph frequently.